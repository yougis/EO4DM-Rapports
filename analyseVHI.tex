% Options for packages loaded elsewhere
\PassOptionsToPackage{unicode}{hyperref}
\PassOptionsToPackage{hyphens}{url}
\PassOptionsToPackage{dvipsnames,svgnames,x11names}{xcolor}
%
\documentclass[
  10pt,
  letterpaper,
  DIV=11,
  numbers=noendperiod]{scrartcl}

\usepackage{amsmath,amssymb}
\usepackage{iftex}
\ifPDFTeX
  \usepackage[T1]{fontenc}
  \usepackage[utf8]{inputenc}
  \usepackage{textcomp} % provide euro and other symbols
\else % if luatex or xetex
  \usepackage{unicode-math}
  \defaultfontfeatures{Scale=MatchLowercase}
  \defaultfontfeatures[\rmfamily]{Ligatures=TeX,Scale=1}
\fi
\usepackage{lmodern}
\ifPDFTeX\else  
    % xetex/luatex font selection
\fi
% Use upquote if available, for straight quotes in verbatim environments
\IfFileExists{upquote.sty}{\usepackage{upquote}}{}
\IfFileExists{microtype.sty}{% use microtype if available
  \usepackage[]{microtype}
  \UseMicrotypeSet[protrusion]{basicmath} % disable protrusion for tt fonts
}{}
\makeatletter
\@ifundefined{KOMAClassName}{% if non-KOMA class
  \IfFileExists{parskip.sty}{%
    \usepackage{parskip}
  }{% else
    \setlength{\parindent}{0pt}
    \setlength{\parskip}{6pt plus 2pt minus 1pt}}
}{% if KOMA class
  \KOMAoptions{parskip=half}}
\makeatother
\usepackage{xcolor}
\usepackage[top=20mm,left=20mm,right=20mm]{geometry}
\setlength{\emergencystretch}{3em} % prevent overfull lines
\setcounter{secnumdepth}{5}
% Make \paragraph and \subparagraph free-standing
\ifx\paragraph\undefined\else
  \let\oldparagraph\paragraph
  \renewcommand{\paragraph}[1]{\oldparagraph{#1}\mbox{}}
\fi
\ifx\subparagraph\undefined\else
  \let\oldsubparagraph\subparagraph
  \renewcommand{\subparagraph}[1]{\oldsubparagraph{#1}\mbox{}}
\fi


\providecommand{\tightlist}{%
  \setlength{\itemsep}{0pt}\setlength{\parskip}{0pt}}\usepackage{longtable,booktabs,array}
\usepackage{calc} % for calculating minipage widths
% Correct order of tables after \paragraph or \subparagraph
\usepackage{etoolbox}
\makeatletter
\patchcmd\longtable{\par}{\if@noskipsec\mbox{}\fi\par}{}{}
\makeatother
% Allow footnotes in longtable head/foot
\IfFileExists{footnotehyper.sty}{\usepackage{footnotehyper}}{\usepackage{footnote}}
\makesavenoteenv{longtable}
\usepackage{graphicx}
\makeatletter
\def\maxwidth{\ifdim\Gin@nat@width>\linewidth\linewidth\else\Gin@nat@width\fi}
\def\maxheight{\ifdim\Gin@nat@height>\textheight\textheight\else\Gin@nat@height\fi}
\makeatother
% Scale images if necessary, so that they will not overflow the page
% margins by default, and it is still possible to overwrite the defaults
% using explicit options in \includegraphics[width, height, ...]{}
\setkeys{Gin}{width=\maxwidth,height=\maxheight,keepaspectratio}
% Set default figure placement to htbp
\makeatletter
\def\fps@figure{htbp}
\makeatother

\usepackage{typearea}
\usepackage{scrlayer-scrpage}
\rofoot{Diffusion limitée - OEIL}
\KOMAoption{captions}{tableheading}
\usepackage{float}
\floatplacement{table}{H}
\makeatletter
\@ifpackageloaded{caption}{}{\usepackage{caption}}
\AtBeginDocument{%
\ifdefined\contentsname
  \renewcommand*\contentsname{Table des matières}
\else
  \newcommand\contentsname{Table des matières}
\fi
\ifdefined\listfigurename
  \renewcommand*\listfigurename{Liste des Figures}
\else
  \newcommand\listfigurename{Liste des Figures}
\fi
\ifdefined\listtablename
  \renewcommand*\listtablename{Liste des Tables}
\else
  \newcommand\listtablename{Liste des Tables}
\fi
\ifdefined\figurename
  \renewcommand*\figurename{Figure}
\else
  \newcommand\figurename{Figure}
\fi
\ifdefined\tablename
  \renewcommand*\tablename{Table}
\else
  \newcommand\tablename{Table}
\fi
}
\@ifpackageloaded{float}{}{\usepackage{float}}
\floatstyle{ruled}
\@ifundefined{c@chapter}{\newfloat{codelisting}{h}{lop}}{\newfloat{codelisting}{h}{lop}[chapter]}
\floatname{codelisting}{Listing}
\newcommand*\listoflistings{\listof{codelisting}{Liste des Listings}}
\makeatother
\makeatletter
\makeatother
\makeatletter
\@ifpackageloaded{caption}{}{\usepackage{caption}}
\@ifpackageloaded{subcaption}{}{\usepackage{subcaption}}
\makeatother
\ifLuaTeX
\usepackage[bidi=basic]{babel}
\else
\usepackage[bidi=default]{babel}
\fi
\babelprovide[main,import]{french}
% get rid of language-specific shorthands (see #6817):
\let\LanguageShortHands\languageshorthands
\def\languageshorthands#1{}
\ifLuaTeX
  \usepackage{selnolig}  % disable illegal ligatures
\fi
\usepackage{bookmark}

\IfFileExists{xurl.sty}{\usepackage{xurl}}{} % add URL line breaks if available
\urlstyle{same} % disable monospaced font for URLs
\hypersetup{
  pdftitle={Analyse de données Sècheresse},
  pdfauthor={Observatoire de l'Environnement en Nouvelle-Calédonie (OEIL)},
  pdflang={fr},
  colorlinks=true,
  linkcolor={blue},
  filecolor={Maroon},
  citecolor={Blue},
  urlcolor={Blue},
  pdfcreator={LaTeX via pandoc}}

\title{Analyse de données Sècheresse}
\author{Observatoire de l'Environnement en Nouvelle-Calédonie (OEIL)}
\date{26 février 2024}

\begin{document}
\maketitle

\renewcommand*\contentsname{Table des matières}
{
\hypersetup{linkcolor=}
\setcounter{tocdepth}{3}
\tableofcontents
}
\section{Evolution des moyennes mensuelles de VHI en
Nouvelle-Calédonie}\label{evolution-des-moyennes-mensuelles-de-vhi-en-nouvelle-caluxe9donie}

Ci-dessous, nous visualisons l'évolution des moyennes mensuelles de VHI
en Nouvelle-Calédonie en rouge. En gris, nous visualisons l'évolution
des moyennes mensuelles de VHI pour chaque commune (superposée). Ce mode
de representation permet de comparer l'évolution de la
Nouvelle-Calédonie avec celle de l'ensemble des communes.

\begin{figure}[H]

{\centering \includegraphics{analyseVHI_files/figure-pdf/cell-4-output-1.pdf}

}

\caption{Evolution des moyennes mensuelles de VHI en Nouvelle-Calédonie}

\end{figure}%

Ci dessous une visualisation de l'évolution temporelle des moyennes de
VHI par commune et par mois. Les valeurs extrêmes sont mises en évidence
pour les communes (bleue) et pour la Nouvelle-Calédonie (rouge).

\begin{figure}[H]

{\centering \includegraphics{analyseVHI_files/figure-pdf/cell-5-output-1.pdf}

}

\caption{Evolution temporelle des moyennes de VHI par commune et par
mois}

\end{figure}%

Zoom sur les valeurs extrêmes de la Nouvelle-Calédonie. Nous percevons
les grandes années de sécheresse. Ce graphique illustre le 5e percentile
de la moyenne sur la Nouvelle-Calédonie, indiquant ainsi la valeur sous
laquelle se situent 5 \% des observations les plus faibles de cette
colonne.

\includegraphics{analyseVHI_files/figure-pdf/cell-6-output-1.pdf}

\begin{longtable}[]{@{}lrr@{}}
\caption{Nombre de valeurs extrêmes par annéegraphiqe}\tabularnewline
\toprule\noalign{}
& ANNEE & Nombre de valeurs extrêmes par année \\
\midrule\noalign{}
\endfirsthead
\toprule\noalign{}
& ANNEE & Nombre de valeurs extrêmes par année \\
\midrule\noalign{}
\endhead
\bottomrule\noalign{}
\endlastfoot
0 & 2001 & 1 \\
1 & 2002 & 3 \\
2 & 2003 & 2 \\
3 & 2004 & 6 \\
4 & 2005 & 2 \\
5 & 2006 & 3 \\
6 & 2007 & 3 \\
7 & 2010 & 2 \\
8 & 2014 & 3 \\
9 & 2016 & 1 \\
10 & 2017 & 8 \\
11 & 2019 & 3 \\
12 & 2020 & 3 \\
\end{longtable}

\section{Détail de l'évolution des moyennes trimestrielles de VHI par
commune}\label{duxe9tail-de-luxe9volution-des-moyennes-trimestrielles-de-vhi-par-commune}

Pour chaque commune, nous traçons l'évolution temporelle des moyennes
trimestrielles de VHI, avec les écarts types, les valeurs minimales et
maximales.

\includegraphics{analyseVHI_files/figure-pdf/cell-8-output-1.pdf}

\includegraphics{analyseVHI_files/figure-pdf/cell-8-output-2.pdf}

\includegraphics{analyseVHI_files/figure-pdf/cell-8-output-3.pdf}

\includegraphics{analyseVHI_files/figure-pdf/cell-8-output-4.pdf}

\includegraphics{analyseVHI_files/figure-pdf/cell-8-output-5.pdf}

\includegraphics{analyseVHI_files/figure-pdf/cell-8-output-6.pdf}

\includegraphics{analyseVHI_files/figure-pdf/cell-8-output-7.pdf}

\includegraphics{analyseVHI_files/figure-pdf/cell-8-output-8.pdf}

\includegraphics{analyseVHI_files/figure-pdf/cell-8-output-9.pdf}

\includegraphics{analyseVHI_files/figure-pdf/cell-8-output-10.pdf}

\includegraphics{analyseVHI_files/figure-pdf/cell-8-output-11.pdf}

\includegraphics{analyseVHI_files/figure-pdf/cell-8-output-12.pdf}

\includegraphics{analyseVHI_files/figure-pdf/cell-8-output-13.pdf}

\includegraphics{analyseVHI_files/figure-pdf/cell-8-output-14.pdf}

\includegraphics{analyseVHI_files/figure-pdf/cell-8-output-15.pdf}

\includegraphics{analyseVHI_files/figure-pdf/cell-8-output-16.pdf}

\includegraphics{analyseVHI_files/figure-pdf/cell-8-output-17.pdf}

\includegraphics{analyseVHI_files/figure-pdf/cell-8-output-18.pdf}

\includegraphics{analyseVHI_files/figure-pdf/cell-8-output-19.pdf}

\includegraphics{analyseVHI_files/figure-pdf/cell-8-output-20.pdf}

\includegraphics{analyseVHI_files/figure-pdf/cell-8-output-21.pdf}

\includegraphics{analyseVHI_files/figure-pdf/cell-8-output-22.pdf}

\includegraphics{analyseVHI_files/figure-pdf/cell-8-output-23.pdf}

\includegraphics{analyseVHI_files/figure-pdf/cell-8-output-24.pdf}

\includegraphics{analyseVHI_files/figure-pdf/cell-8-output-25.pdf}

\includegraphics{analyseVHI_files/figure-pdf/cell-8-output-26.pdf}

\includegraphics{analyseVHI_files/figure-pdf/cell-8-output-27.pdf}

\includegraphics{analyseVHI_files/figure-pdf/cell-8-output-28.pdf}

\includegraphics{analyseVHI_files/figure-pdf/cell-8-output-29.pdf}

\includegraphics{analyseVHI_files/figure-pdf/cell-8-output-30.pdf}

\includegraphics{analyseVHI_files/figure-pdf/cell-8-output-31.pdf}

\includegraphics{analyseVHI_files/figure-pdf/cell-8-output-32.pdf}

\includegraphics{analyseVHI_files/figure-pdf/cell-8-output-33.pdf}

\section{Evolution de la moyenne mensuelle des QSCORE par
mois}\label{evolution-de-la-moyenne-mensuelle-des-qscore-par-mois}

QSCORE est un indice moyen historique estimé pour chaque mois et chaque
décade qui évalue la qualité des données valides utilisées dans le
calcul des anomalies. Il s'agit ici de prendre en compte la propagation
des données manquantes (ou à l'inverse ``valides'') issues de
l'historiques (série temporelle) dans l'indicateur de sécheresse final.

\includegraphics{analyseVHI_files/figure-pdf/cell-9-output-1.pdf}



\end{document}
